\section{Dealing with samples and patients}
We will see how to make the best use of the patient and sample database and
how to use it efficiently.
For this sake you need an account with the rights to create new patients,
runs, sets, to upload data and, preferably, to run analyses.
Therefore the demo account is not suitable.

\question{ Retrieve the toy dataset at
  \href{http://vidjil.org/seqs/tutorial_dataset.zip}{vidjil.org/seqs/tutorial\_dataset.zip}
  and extract the files from the archive.}

You should now have three files. We will imagine that those three files are
the results from a single sequencing run. More precisely, each one corresponds to
a single patient. Thus we now want to upload those files and assign all of
them to a same \com{run} and each of them to a single \com{patient}.

\question{
  Go to the main page of the Vidjil platform (by default
  \href{https://app.vidjil.org}{app.vidjil.org}).
  You should be on the \com{patients} page.
  Go at the bottom of the page and click on \com{+ new patients} to create the
  three patients.
}

Note that usually you should check whether the patient has
already been created by searching her/his name in the search box at the
upper left corner

\question{
  You are now on the  creation page for patients, runs, and sets.
  You can create as many patients, runs and sets as you want.
  \marginpar{Patients, runs and sets are just different ways to
    group samples. 
    The names are just used to add some semantic so that you
    know that your patients will be on the patient page, your runs on the run
    page and your other sets (thus any set of samples you want to make) on the
    run page.}
  Here we already have a line to create one patient.
  We want to create two additional patients and one run.
  Thus click twice on \com{add patient} and once on \com{add run}.
}

  Now you should have three lines with Patient 1, Patient 2, Patient 3 and one
  line with Run 1.
  If you created too many lines you can remove some by clicking on the cross at
  the right hand side.

\question{
  For instance click on the cross corresponding to Patient 3.
  The line has now been removed.
  Click again on \com{add patient} so that the line appears again (it is now
  called Patient 4).
}

\question{ Now you can fill the mandatory fields (circled with red) and,
  optionally, the other fields.}

The last field is optional but it is very important (the field called
\com{patient/run information (\#tags can be used)}.

Here you can enter any information relevant to this set of samples.
More specifically you can enter tags (starting with a \#) that will allow
you to search very easily and quickly all the patients/runs/sets sharing
this tag.
By default when you enter a \# in this field, some tags appear and the
suggestions are  updated while you enter other characters.
Note that a tag cannot contain any space.
Also note that you can create other tags just by entering whatever you would
like in the field preceded with a \#. Thus any tag you enter is saved (and
can be suggested later on).

\question{For patient 1 in this last field, enter \texttt{\#diagnosis of
    patient with \#B-ALL}. For patient 2, enter \texttt{\#blood sample \#CLL}.
  For patient 4, enter \texttt{bone \#marrow \#B-ALL}}

Now the three patients and the run have been created but we have not uploaded
the sequence files yet.

\question{Now go to the \com{runs} page. You should see the run you have just
  created. Click on it. Then click on \com{+ add samples}.}

Similarly to the patient/run creation page, we can add as many samples as we
want on this page.

\question{As we need to upload three samples, click twice on the \com{add
    other sample} button so that you have three lines to add a sample.}

\question{For sample 1, choose the file corresponding to patient 1 (and
  respectively for patient 2 and 3). You can also add extra information, with
  tags, as previously.}

Note the \com{common sets} field. This field means that all the samples will
be added to this run (the one you created). If you would like to \textbf{all}
the samples to another patient/run/set you should specify it here.

In our case we want to add each sample to a different patient. Thus we don't
need to modify this field.

\question{Instead we need to modify the last field on each line. Click on
  it. A list should appear with the last patients/runs/sets you created.
  Either click on the correct patient or type the first letters of her/his
  name. Then validate with \com{Enter} or by clicking on the correct entry.}

\question{When you have associated each sample to its corresponding patient
  you can upload the samples by clicking on the \com{Submit samples} button.}

Now you are back on the page of the run where you should see the three samples
that are being uploaded.

\question{When the upload is finished you launch the analysis by selecting the
configuration in the drop down at the right (\com{multi+inc+xxx}) and then
clicking on the gearwheel.}

You can have a coffee, a tea, or something else, while the process is
launched.

\question{To regularly check the status of your job you can click
  on the \com{reload} button at the button left of the page.
  Your process usually goes through the following stages: \com{QUEUED},
  (possibly \com{STOPPED}), \com{ASSIGNED}, \com{RUNNING}, \com{COMPLETED} (or
  \com{FAILED} when there is an issue, in such a case please contact us)}

Then you can view the results as explained before.
Instead we will remain on the server.

\question{Now go back to the \com{patients} page. You can filter the page
  using the tags you entered previously.
  Enter \texttt{\#B-ALL} in the search box (notice the autocompletion that helps you)
  and validate with \com{Enter}.}
